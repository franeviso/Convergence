%% Template for a preprint Letter or Article for submission
%% to the journal Nature.
%% Written by Peter Czoschke, 26 February 2004
%%

\documentclass[pdftex]{nature}

%% make sure you have the nature.cls and naturemag.bst files where
%% LaTeX can find them

%\bibliographystyle{naturemag}

%\title{Frequency-dependent selection predicts patterns of radiations and biodiversity}

%% Notice placement of commas and superscripts and use of &
%% in the author list

%\author{Carlos J. Meli\'an$^1$, David Alonso$^2$ \& Diego P. V\'azquez$^{3,4}$, James Regetz$^{1}$ \& and Stefano Allesina$^{1}$}


\begin{document}
\parskip 6pt
\baselineskip 12pt
\noindent

\thispagestyle{empty}
Dear Editor, dear members of the editorial board, 
\vspace{0.1 in}

Please find enclosed the submission of our manuscript
``Eco-evolutionary diversification of trait convergence and
complementarity in mutualistic networks'' to Ecology Letters.

In recent years, a great deal of progress towards understanding
convergence, complementarity and nestedness in mutualistic networks has been
made. Empirical mutualistic networks composed by several interacting
species show high levels of trait convergence, complementarity and
nestedness. Yet theories with co-evolutionary selection aiming to
predict trait convergence, complementarity and nestedness suggest that
there may be strong trade-offs between these variables that limit model
predictions to match the empirical observations. For example, for weak
or absent coevolutionary selection, trait values in animal and plant
species can be highly variable and non-convergent but trait values of
animal and plants species are positively correlated (i.e.,
complementarity). As coevolutionary selection intensifies, variation
in the trait values of animal and plant species is reduced and
converge but correlations between traits of interacting species are
weakened (i.e., low pairwise complementarity). Whether co-evolutionary
selection is required to explain trait convergence, complementarity
and nestedness, the impact of population and diversification dynamics
on quantitative trait divergence and convergence dynamics in
species-rich mutualistic networks remains largely unexplored. 

Here, we present a landscape genetics model to connect population and
diversification dynamics to quantitative trait dynamics to study trait
complementarity and convergence in species-rich mutualistic
networks. After controlling by phylogenetically relatedness, our
results show that population and diversification dynamics drive
convergence and complementarity by producing a gradient of species
phenotypes ranging from common and large trait variation plant and
animal species to rare species with small trait variation. Our results
also predict convergence values largely independent or positively
correlated to trait complementarity and nestedness between plant and
animals. Our model is in agreement with observed levels of
complementarity and convergence for animals, but conflicts with
convergence patterns for plants in an empirical plant-hummingbird
mutualistic network. Our results indicate that combining the spatial
structure of plant-animal trait diversification with population
dynamics in a phenotypic difference model alter predictions in a
direction that is more in line with observations, and hence are
important to be included in future studies of evolutionary patterns in
mutualistic networks.

Taken togehter, our results show no signs of trade-offs between
convergence, complementarity and nestedness and suggest that
diversification trait dynamics requires ecological, genetical and
morphological processes associated to phenotype differences to
reproduce key patterns of mutualistic networks, from trait convergence
and complementarity to connectance and nestedness.

All the authors confirm these results are outstandingly novel relative
to current co-evolutionary theory of mutualistic networks and to
recent work by coauthors (pdf's attached with the submission).

\noindent
We thank you in advance for the attention dedicated to this
manuscript.


\noindent
Yours sincerely,

\noindent
Francisco Encinas-Viso\\
Rampal S. Etienne\\
Carlos J. Meli\'an\\

\baselineskip 24pt
\parskip 12pt
\end{document}
